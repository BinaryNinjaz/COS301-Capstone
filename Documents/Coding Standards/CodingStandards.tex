\documentclass[12pt]{article}
\newcommand\tab[1][0.5cm]{\hspace*{#1}}
\usepackage{graphicx}

\usepackage{hyperref}
\hypersetup{
  colorlinks=true,
  linkcolor=black,
  filecolor=magenta,
  urlcolor=cyan,
}
\begin{document}

\begin{titlepage}

	\begin{center}
		% Upper part of the page
		\includegraphics[height=5em]{images/Bin.png}\\
		\textsc{\LARGE Binary Ninjaz}\\[0.3cm]
		% Title
		\rule{\linewidth}{0.5mm} \\[0.5cm]
		\includegraphics[height=5em]{images/Icon.png}\\
		{ \huge \bfseries Harvest \\
		  \vspace{0.3cm}\large \bfseries Coding Standards}\\[0.5cm]
		\rule{\linewidth}{0.5mm} \\[1cm]


		\begin{minipage}{0.4\textwidth}
			\begin{flushleft} \large
				\emph{} \\
				Letanyan {Arumugam}
			\end{flushleft}
		\end{minipage}
		\begin{minipage}{0.4\textwidth}
			\begin{flushright} \large
				\emph{} \\
				14228123
			\end{flushright}
		\end{minipage}

		\begin{minipage}{0.4\textwidth}
			\begin{flushleft} \large
            	\emph{} \\
				Sizo {Duma}
			\end{flushleft}
		\end{minipage}
		\begin{minipage}{0.4\textwidth}
			\begin{flushright} \large
				\emph{} \\
				15245579
			\end{flushright}
		\end{minipage}

		\begin{minipage}{0.4\textwidth}
			\begin{flushleft} \large
				\emph{} \\
				Teboho {Mokoena}
			\end{flushleft}
		\end{minipage}
		\begin{minipage}{0.4\textwidth}
			\begin{flushright} \large
				\emph{} \\
				14415888
			\end{flushright}
		\end{minipage}

		\begin{minipage}{0.4\textwidth}
			\begin{flushleft} \large
				\emph{} \\
				John {Ojo}
			\end{flushleft}
		\end{minipage}
		\begin{minipage}{0.4\textwidth}
			\begin{flushright} \large
				\emph{} \\
				15096794
			\end{flushright}
		\end{minipage}

        \begin{minipage}{0.4\textwidth}
			\begin{flushleft} \large
				\emph{} \\
				Kevin {Reid}
			\end{flushleft}
		\end{minipage}
		\begin{minipage}{0.4\textwidth}
			\begin{flushright} \large
				\emph{} \\
				15008739
			\end{flushright}
		\end{minipage}

        \begin{minipage}{0.4\textwidth}
			\begin{flushleft} \large
				\emph{} \\
				Shaun {Yates}
			\end{flushleft}
		\end{minipage}
		\begin{minipage}{0.4\textwidth}
			\begin{flushright} \large
				\emph{} \\
				16007493
			\end{flushright}
		\end{minipage}

		\vspace{1cm}
		\rule{\linewidth}{0.5mm} \\[1cm]
		\textsc{\Large Stakeholders}\\[1cm]

		\begin{minipage}{0.4\textwidth}
			\begin{flushleft} \large
				\emph{} \\
				SAMAC:
			\end{flushleft}
		\end{minipage}
		\begin{minipage}{0.4\textwidth}
			\begin{flushright} \large
				\emph{} \\
				Barry Christie
			\end{flushright}
		\end{minipage}


	\end{center}
\end{titlepage}

\newpage
\pagenumbering{Roman}
\tableofcontents
\newpage
\listoffigures

\newpage
\pagenumbering{arabic}

\begin{figure*}
\includegraphics[width=1.0\linewidth]{UMLClassDiag.png}
\caption{UML Class Diagram}
\label{UMLClassDiag}
\end{figure*}
\newpage

\section{Coding Conventions}
  \flushleft
  The project and design we've undertaken means that we will need to use multiple languages and IDE's to complete the project. As such we have decided that the best common practice and standards of each language shall be the standard that is used for that respective language/environment. Linters are also used to enforce many of the rules we follow.
  \subsection{General Rules}
  \begin{itemize}
  \item Attributes:\newline
    special @ sign attributes should be on their own line. \newline
     \newline
  \item Class Delegate Pattern:\newline
    Delegate pattern should be implemented using protocols/interfaces. In the case of protocols \newline
    the protocol must be a class protocol. \newline
     \newline
  \item Closing Brace:\newline
    The closing brace inside a parameter must not be followed by whitespace. \newline
    Incorrect Usage: `array.map({ .0 }  )` \newline
    Correct Usage: `array.map({ .0 })` \newline
     \newline
  \item Closure Closing Indentation:\newline
    The closures closing brace must end on the same column that the starting lines column is on. \newline
    Incorrect Usage: \newline
    \begin{verbatim}
    func foo() {
      }
    \end{verbatim}
    Correct Usage: \newline
    \begin{verbatim}
    func foo() {
    }
    \end{verbatim}
  \item Closure Spacing:\newline
    Closures must have at least a single space inside of each brace. \newline
     \newline
  \item Comma Spacing:\newline
    A comma must not have any space infront of it and only one space after it. \newline
     \newline
  \item Conditional Return Positions:\newline
    Any return statement from a conditional branch must be on a newline \newline
    Incorrect Usage: \newline
    \begin{verbatim}
    if (true) { return 0; } else { return 1; }
    \end{verbatim}
    Correct Usage: \newline
    \begin{verbatim}
    if (true) {
      return 0;
    } else {
      return 1;
    }
    \end{verbatim}
  \item Cyclomatic Complexity:\newline
    The cyclomatic complexity of a function body should not exceed more than 10. \newline
     \newline
  \item Singleton Direct Initialization:\newline
    Singleton classes should not be directly instantiated. \newline
     \newline
  \item Optional Boolean:\newline
    Where available an Optional may not be used to wrap a Boolean. Prefer an enum instead. \newline
     \newline
  \item Prefer Non-Optional Collection:\newline
    Where available an Optional should not be a preferred method of wrapping a collection. Instead try to \newline
    use a collections emptiness to handle nullity as well. \newline
     \newline
  \item Empty Count:\newline
    Use isEmpty property and not check emptiness via count/length property. \newline
     \newline
  \item Empty Parameter:\newline
    Prefer Void over an empty parameter \newline
     \newline
  \item String isEmpty:\newline
    Prefer using isEmpty over checking equality to an empty string \newline
     \newline
  \item Avoid Fallthrough:\newline
    Switch statements should not fallthrough to subsequent cases but instead always break at the end of  \newline
    a case. \newline
     \newline
  \item File Length:\newline
    File lengths should not exceed more than 2000 lines. \newline
     \newline
  \item File Name:\newline
    A file name should match the name of a class/struct/extension it contains. \newline
     \newline
  \item Function Body Length:\newline
    Length of a function body should not exceed 150 lines. \newline
     \newline
  \item Default Parameters at End:\newline
    Default parameters should be the last parameters in a parameter list. \newline
     \newline
  \item Function Parameter Count:\newline
    A maximum of 5 parameters per function should be allowed. \newline
     \newline
  \item Type Names:\newline
    Type names should start with an uppercase letter and be of length 1 to 20 characters. Each \newline
    character should only be an alphanumeric. Where subsequent words in the name are also  \newline
    capitalized. \newline
     \newline
  \item Identifier Name:\newline
    Type names should start with an lowercase letter and be of length 1 to 20 characters. Each \newline
    character should only be an alphanumeric. Where subsequent words in the name are capitalized. \newline
    Or names can be all uppercased if deemed fit. \newline
     \newline
  \item Tuple Size:\newline
    Where available tuple sizes may not exceed an arity of 3. \newline
     \newline
  \item Leading Whitespace:\newline
    Files may not start with whitespace (excluding comments) \newline
     \newline
  \item Line Length:\newline
    Lines should not exceed 120 characters. \newline
     \newline
  \item Mark Comments:\newline
    Mark comments should follow "MARK: ..." or "MARK: -..." \newline
     \newline
  \item Fix Me Comments:\newline
    Fix me comments should be treated as warnings and follow the format "FIXME: ..." \newline
  \item Todo Comments:\newline
    Todo comments should be treated as warnings and follow the format "TODO: ..." \newline
  \item Type Nesting:\newline
    Types should not be nested more than one type scope deep. \newline
     \newline
  \item Open Brace Spacing:\newline
    Open braces should be on the same line as the declaration and be preceded by a single space. \newline
     \newline
  \item Operator Whitespace:\newline
    Operators should be surrounded by an equal number of whitespace characters. \newline
     \newline
  \item Redundant Void Return:\newline
    Void returning functions should not be explicitly stated if not needed. \newline
     \newline
  \item Modify Assign Operators:\newline
    Prefer operators such as +=, *=, -=, /= over constructs such as `a = a + 42` \newline
     \newline
  \item Sorted Imports:\newline
    Import list should be sorted alphabetically \newline
     \newline
  \item Statement Positions:\newline
    Else and catch statements should be on their own line. \newline
     \newline
  \item Switch Case Statement Indentation:\newline
    Case statement indentation should be the same as the switch statements indentation. \newline
    Incorrect Usage: \newline
    \begin{verbatim}
    switch f {
      case a:
    }
    \end{verbatim}
    Correct Usage: \newline
    \begin{verbatim}
    switch f {
    case a:
    }
    \end{verbatim}
  \item Trailing Newline:\newline
    Files should have a single newline at the end. \newline
     \newline
  \item Type Body Length:\newline
    A types body length should not exceed 500 lines. \newline
     \newline
  \item Unneeded Break:\newline
    Switch statements should not have unnecessary break statements. \newline
     \newline
  \item Vertical Whitespace:\newline
    Limit vertical whitespace to a single empty line. \newline
  \end{itemize} \newpage
  \subsection{Conventions used for Specific Languages}
  \subsubsection{HTML/CSS}
  The HTML and CSS standards we shall follow will be the ones of \href{https://google.github.io/styleguide/htmlcssguide.html}{Google's HTML/CSS style guide}

  \subsubsection{JavaScript}
  For JavaScript we will be using \href{https://google.github.io/styleguide/jsguide.html}{Google's JavaScript style guide}.

  \subsubsection{Java}
  Similarly we will use \href{https://google.github.io/styleguide/javaguide.html}{Google's Java style guide}.

  \subsubsection{Swift}
  For Swift we will use \href{https://swift.org/documentation/api-design-guidelines/}{Apple design guidelines}.

\section{File Structure}
  \flushleft
  \begin{figure*}
  \includegraphics[width=1.2\linewidth]{FileStructure.png}
  \caption{File Structure}
  \label{FileStructure}
  \end{figure*}

  It is important to note that \texttt{Web/app} is where the main functionality of the website source code can be found and that \texttt{Web/functions} is strictly used for cloud functions.

  \begin{figure*}
  \includegraphics[width=1.2\linewidth]{GitBranchStructure.png}
  \caption{Git Branch Structure}
  \label{GitBranchStructure}
  \end{figure*}

\flushleft\subsubsection{Branching}
  An example structure is shown above, showing that \texttt{master} is never worked on directly. It is important to note that branch names given of feature branches are not necessarily set branches, but are examples to show branching structure. Each subsystem (Android, iOS, Website) has its own respective branch, and these branches branch off \texttt{Developer}, and in turn any changes done to a branch would then branch off that specific branch.
  A \texttt{Developer} branch is used, to act in much the same way as \texttt{master}, but as a proxy, so that before merging that branch to master, the entire system can be reviewed there before a more solid commitment is made.
  \subsubsection{Naming}
  \flushleft
  A branch shall never contain an individuals name, rather it must be informative as to what the intention of the branch is whilst also remaining short. An example of a  name that would be deemed unacceptable is \texttt{Joe}; another would be \texttt{JoeWebsiteChanges}, where the name contains a member name, and gives very little information as to what is going on in the branch. Bad, but barely acceptable names are \texttt{WebsiteChanges}, or \texttt{WebsiteExperimentation}. Ideal names are as follows: in the case of a branch where the website is ultimately being assembled, by merging other branches into it, and the only modifications taking place are to structure the files, or a quick fix, such as changing the colour of a button, \texttt{Website}; in the case that a new homepage is created for the website, where the homepage is created, and merged into \texttt{Website} on completion, \texttt{Website-Homepage}. Finally, if a user wants to propose a change to the \texttt{Website-Homepage} branch, where he wants to change the colour of a button, but the user has not been assigned, or is not the primary benefactor to \texttt{Website-Homepage}, he creates a new \texttt{Website-Homepage-ButtonColourChange} branch. Notice that in the ideal examples it is possible to track down the source of a branch, to understand exactly what it's purpose is.

  \newpage
  \center\section{Code Review Process}
  \flushleft
All code reviewing is done on GitHub, and occurs every time there are changes done to code, or if new code has been implemented. Currently, pre-commit reviews are utilized and although it is a slower process compared to post-commit reviews, it always for more people to be aware of changes to the code and could possibly lead to better implementation as more eyes usually means more ideas. \newline
After changes have been made on the branch, the creator, who wants to merge, will then make a pull request on GitHub, and announce the request on Slack. After which it shall become the responsibility of the entire team, but more importantly that of the original branch maintainer to analyze and comment on the changes. The exact location of the discussion is trivial, however GitHub provides a better, and more concrete platform for this discussion to take place, where inline comments can be made. After a sufficient amount of time and discussion has passed, either the branch maintainer (if it is a minor change), or the entire team (if it is to master, or a major change) shall decide to merge or not.


\end{document}
