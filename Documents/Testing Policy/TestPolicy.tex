\documentclass[12pt]{article}
\title{Testing Policy}
\author{Binary Ninjaz}
\date{}

\begin{document}
  \maketitle
  \newpage
  
  \tableofcontents
  \newpage
  
  \section{INTRODUCTION}
  This document details the policies that BinaryNinjaz will be using to guide development and ensure a quality product. We describe why we intend to use specific testing methods and how their effects will impactful in our process. 
  
  \section{PURPOSE}
  Automated testing is vital and ensuring that the number of bugs in a program is as limited as possible. While having automated testing does not guarantee there will be no bugs in a program it can greatly reduce any bugs from appearing when you might not expect them to. While we intend to have a comprehensive test suite, we will not accept that wholly passing test cases mean that a program is without flaw.

\section{WHY IS TESTING REQUIRED?}
Testing forms an essential precondition to ensure the successful construction and implementation of information systems. The complexity of modern-day software is such that it is almost impossible to implement it correctly the first time around, without any form of verification.\newline\newline Testing is needed in order to detect potential problems within the software as early as possible, so that they can be corrected at minimum cost.\newline\newline  A second reason to carry out tests is to develop trust in and a knowledge of the product provided.\newline\newline Defects that exist within a software product can have severe consequences for the “business” and the users alike.  Whilst providing a means of avoiding faults as much as possible, testing is also a useful way of demonstrating to management and users that the product supplied fulfils their requirements (is “fit for purpose”).\newline\newline It is important to note in this regard that both the functionality and the non-functional software characteristics play a significant part in asserting that a product fulfils the stated requirements and is useable in its operational context. 

\section{WHAT IS TESTING?}
Testing software takes the form of a process that is used to verify and validate that a software program, application or product: 
\begin{enumerate}
	\item Fulfils the business and technical requirements set out in the contract documents, the requirements, the analysis and design documents
	\item Works as expected and can be used within its operational environment  
	\item Has been implemented with the required non-functional software characteristics 	
\end{enumerate}

\section{BASIC PRINCIPLES}
A number of principles apply to all forms of testing: \newline
 \subsection{Principle 1: Testing reveals defects.}
Testing reveals defects that are present, but is unable to provide evidence that no defects are present. Testing reduces the likelihood that the software contains undiscovered defects, but if no defects are found, this cannot be regarded as proof that no defects are present. \newline
 \subsection{Principle 2: Exhaustive testing is impossible.}
Comprehensive  testing  (all  combinations  of  inputs/outputs  and  preconditions)  is  not feasible, except in trivial cases. Instead of carrying out extensive testing, risk analyses and priorities must be used in  order to  restrict the effort involved in  carrying out  tests to the tests that genuinely need to be carried out.  \newline
 \subsection{Principle 3: Test at an early stage. }
The  testing  activities  must  begin  as  early  as  possible  within  the  software  development cycle. This will ensure that the defects are detected at an early stage, with the result that rectifying the defects will be less costly. \newline
 \subsection{Principle 4: Clustering of defects. }
A small number of the modules contain the largest number of defects discovered during the pre-release tests and/or are responsible for the most operational errors. \newline
  \subsection{Principle 5: The pesticides paradox. }
If the same set of test cases are carried out once again each time, there will come a time when they no longer reveal any defects. That is the reason why the test cases need to be re-examined on a regular basis. New tests must be written in order to verify different parts of the software, so that new defects may be discovered. \newline
\subsection{Principle 6: Testing is context-dependent. }
The test method employed will depend on the context in which the product will ultimately be used. For example, mobile apps will be tested in a different way to a website. \newline
\subsection{Principle 7: The absence-of-errors fallacy. }
Tracing and rectifying defects will be of no benefit if the system is unusable  and/or does not fulfil the needs and expectations of the end-users. \newline

  \section{TESTING TOOLS}
  \subsection{Android Studio}
  The JUnit framwork will be used for out automated testing purposes. To run tests one will need Android Stuio. To execute the tests one can right click on the tests directory and select 'Run tests.'
  \subsection{NPM}
  The Ava framework will be used for automated testing purposes. To run tests one will need to run "npm test" from the testing directory.
  \subsection{XCode}
  The XCTest framework will be used for out automated testing purposes. To run tests one will need XCode. To execute the tests one will need to use XCodes build command.
  
  \section{TESTING PHILOSOPHY}
  Implementation of both simple and holistic tests will be of concern. However, holistic testing will be of more interest. Testing workflows are of vital importance. The project does not have many simple tasks that could provide us with the safety guarantees we would like. We instead seek to find safety in implementing tests that check the whole interaction of these few unit cases.
  
  \subsection{Benefits}
  Automated test provide developers with the ability to have safe of mind that they have not made any changes that break the program in any way or introduce bugs in any part of the program.
  Forcing developers to write tests forces them to check every part of code they write making tests a form of documentation of a projects requirements.
  
  \section{Test Evaluation}
  Strict zero test case failure will be followed. Any branch with a failing test case will never be merged into master. Hence we will ensure that every branch runs tests before a pull request.
  
\end{document}